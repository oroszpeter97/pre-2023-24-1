\documentclass[14pt, aspectratio=169]{beamer}
\usetheme{Copenhagen}
\usecolortheme{beaver}

\setbeamertemplate{navigation symbols}{}

\title{2D Rogue-Like}
\author{Team Cherry}
\date{2022/10/13}

\begin{document}
	\maketitle
	\begin{frame}{Csapattagok}
		\begin{itemize}
			\item Orosz Péter
			\item Dobai Attila
			\item Drahos Alinka
			\item Tőzsér Zétény
			\item Gáncsos Dániel
		\end{itemize}
	\end{frame}
	
	\begin{frame}{Projekt Témája}
		A projekt egy 2d rogue-like játék elkészítése. Maga a játék stílusa pixel art formájában fog megvalósulni és egy kozépkori fantasy világban fog játszódni.
	\end{frame}
	
	\begin{frame}{Programozói nyelvek, fejlesztői eszközök}
		\begin{block}{Programozási Nyelv}
			A program a Java programozási nyelven fog készülni Objektum Orientált alapelveket figyelembe véve, de adatok tárolására a JSON-t fogjuk használni.
		\end{block}
		
		\begin{block}{Külsős Programok}
			\begin{itemize}
				\item Eclipse IDE
				\item Aseprite (Pixel Art -hoz)
				\item BeepBox (A játék zenéjéhez)
			\end{itemize}
		\end{block}
	\end{frame}
	
	\begin{frame}{Játék Célja}
		A játékban több karakter közül lehetlesz választani mindegyik egyedi fejlesztési útvonallal. A játék célja hogy minnél messzeb jusson a játékos és minnél több pontot gyűjtsön össze. Az újrajátszhatóság érdekében a pályák procedúrálisan lesznek generálva. 
	\end{frame}		
	
	\begin{frame}{Végtermék}
		Célunk ezzel a játékkal hogy egy olyan szórakozási lehetőséget hozzunk a játék piacra amit mindenki fog tudni élvezni életkortól függetlenül, hiszen a játék besorolása is PEGI12 lesz.
	\end{frame}
\end{document}